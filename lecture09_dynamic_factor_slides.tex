\documentclass[handout]{beamer}  

%Smaller gap at between top and bottom of block when there are displayed equations
\addtobeamertemplate{block begin}{\setlength\abovedisplayskip{0pt}}
{\setlength{\belowdisplayskip}{0pt}}


\usepackage{setspace}
\linespread{1.2}
\usepackage{amssymb, amsmath, amsthm} 
\usepackage{rotating}
\usepackage{multirow}
\usepackage{graphicx}
\usepackage{synttree}
\usepackage{verbatim}
\usepackage{fancybox}
\usepackage{color}
\usepackage{tikz}
\usetikzlibrary{shapes,backgrounds}
\usepackage{hyperref}
\usetikzlibrary{trees}
\newcommand{\p}{\mathbb{P}}
\newcommand{\expect}{\mathbb{E}}


%\setbeamertemplate{blocks}[rounded][shadow=true] 
%gets rid of bottom navigation bars
\setbeamertemplate{footline}{
   \begin{beamercolorbox}[ht=4ex,leftskip=0.3cm,rightskip=0.3cm]{author in head/foot}
%    \usebeamercolor{UniBlue}
    \vspace{0.1cm}
    %\insertshorttitle \ - \insertdate 
    \hfill \insertframenumber / \inserttotalframenumber
   \end{beamercolorbox}
   \vspace*{0.1cm}
} 


%gets rid of navigation symbols
\setbeamertemplate{navigation symbols}{}


%Include or exclude the notes?
%\setbeameroption{show notes}
\setbeameroption{hide notes}

\title[Econ 722]{Dynamic Factor Models} 
\author[F. DiTraglia]{Francis J.\ DiTraglia}
\institute{University of Pennsylvania}
\date{Econ 722}


\begin{document} 
%%%%%%%%%%%%%%%%%%%%%%%%%%%%%%%%%%%%%%%%



\begin{frame}[plain]
	\titlepage 
	

\end{frame} 


%%%%%%%%%%%%%%%%%%%%%%%%%%%%%%%%%%%%%%%%
 \begin{frame}
 	\frametitle{Last Time: Classical Factor Analysis Model}

 $$\underset{(N\times 1)}{X_t} &=& \underset{(N\times 1)}{\mu} +\underset{(N\times k)}{\Lambda} \underset{(k\times 1)}{Z_t} +\underset{(N\times 1)}{\epsilon_t}$$

 \vspace{2em}

 \small
$$\left[\begin{array}
	{c} Z_t \\ \epsilon_t
\end{array} \right] \overset{iid}{\sim} \mathcal{N} \left(\left[ \begin{array}
	{c} 0_k \\ 0_N
\end{array}\right], \left[ \begin{array}
	{cc} I_k & 0 \\
	0 & I_N
\end{array}\right]\right)$$


 \end{frame}


%%%%%%%%%%%%%%%%%%%%%%%%%%%%%%%%%%%%%%%%
\begin{frame}
\frametitle{What Can We Do with Factors?}

There are just a few possibilities:
\begin{enumerate}
	\item Use them as Instumental Variables 
	\item Use them to construct Forecasts
	\item Use them to ``Augment'' a VAR
\end{enumerate}

\end{frame}

%%%%%%%%%%%%%%%%%%%%%%%%%%%%%%%%%%%%%%%%
\begin{frame}[c]\frametitle{``IV Estimation in a Data Rich Environment'' (Bai \& Ng, 2010)}
\begin{block}
   	{Endogenous Regressors $x_t$}
$$y_t = x_t' \beta + \epsilon_t \quad \quad E[x_t\epsilon_t] \neq 0 $$
\end{block}   

 \begin{block}
 	{Unobserved Variables $F_t$ are Strong IVs}
$$\underset{(k\times 1)}{x_t} = \underset{(k\times r)}{\Psi'}\underset{(r\times 1)}{F_t} + \underset{(k\times 1)}{u_t} \quad \quad E[F_t \epsilon_t] = 0$$
 \end{block}

\begin{block}
	{Observe Large Panel $z_{1t}, \hdots, z_{Nt}$}
		$$z_{it} = \lambda_i' F_t + e_{it}$$
\end{block}

\end{frame}

\end{document}