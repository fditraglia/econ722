\documentclass[handout]{beamer}  

%Smaller gap at between top and bottom of block when there are displayed equations
\addtobeamertemplate{block begin}{\setlength\abovedisplayskip{0pt}}
{\setlength{\belowdisplayskip}{0pt}}


\usepackage{setspace}
\linespread{1.2}
\usepackage{amssymb, amsmath, amsthm} 
\usepackage{rotating}
\usepackage{multirow}
\usepackage{graphicx}
\usepackage{synttree}
\usepackage{verbatim}
\usepackage{fancybox}
\usepackage{color}
\usepackage{tikz}
\usetikzlibrary{shapes,backgrounds}
\usepackage{hyperref}
\usetikzlibrary{trees}
\newcommand{\p}{\mathbb{P}}
\newcommand{\expect}{\mathbb{E}}


%\setbeamertemplate{blocks}[rounded][shadow=true] 
%gets rid of bottom navigation bars
\setbeamertemplate{footline}{
   \begin{beamercolorbox}[ht=4ex,leftskip=0.3cm,rightskip=0.3cm]{author in head/foot}
%    \usebeamercolor{UniBlue}
    \vspace{0.1cm}
    %\insertshorttitle \ - \insertdate 
    \hfill \insertframenumber / \inserttotalframenumber
   \end{beamercolorbox}
   \vspace*{0.1cm}
} 


%gets rid of navigation symbols
\setbeamertemplate{navigation symbols}{}


%Include or exclude the notes?
%\setbeameroption{show notes}
\setbeameroption{hide notes}

\title[Econ 722]{Dynamic Factor Models} 
\author[F. DiTraglia]{Francis J.\ DiTraglia}
\institute{University of Pennsylvania}
\date{Econ 722}


\begin{document} 
%%%%%%%%%%%%%%%%%%%%%%%%%%%%%%%%%%%%%%%%



\begin{frame}[plain]
	\titlepage 
	

\end{frame} 


%%%%%%%%%%%%%%%%%%%%%%%%%%%%%%%%%%%%%%%%
\begin{frame}
\frametitle{Survey Articles on Dynamic Factor Models}

\begin{block}
	{Stock \& Watson (2010)}
	Best general overview of dynamic factor models and applications.
\end{block}

\begin{block}
	{Bai \& Ng (2008)}
Comprehensive review of large-sample results for high-dimensional factor models estimated via PCA.
\end{block}

\begin{block}
	{Stock \& Watson (2006)}
	Handbook chapter on forecasting with many predictors. One section is devoted to dynamic factor models.
\end{block}

\begin{block}
	{Breitung \& Eickmeyer (2006)}
	Brief overview with an application to Euro-area business cycles. 
\end{block}

\end{frame}
%%%%%%%%%%%%%%%%%%%%%%%%%%%%%%%%%%%%%%%%
\begin{frame}[c]\frametitle{Why Factor Models?}
   
\begin{enumerate}
	\item Factors could be intrinsically interesting if they arise from a theoretical model (e.g.\ Financial Economics)
	\item Many variables without running out of degrees of freedom\begin{itemize}
			\item More information could improve forecasts/macro analysis
			\item Mimic central banks ``looking at everything'' 
		\end{itemize}
	\item Eliminate measurement error and idiosyncratic shocks to provide more reliable information for policy
	\item ``Remain Agnostic about the Structure of the Economy''\begin{itemize}
		\item Advantages over SVARs: don't have to choose variables to control degrees of freedom, and can allow fewer underlying shocks than variables. 
	\end{itemize}
\end{enumerate}


\end{frame}

%%%%%%%%%%%%%%%%%%%%%%%%%%%%%%%%%%%%%%%%
\begin{frame}[c]\frametitle{Last Time: Classical Factor Analysis Model}
    
$$\underset{(N\times 1)}{X_t} = \mu + \Lambda \underset{(k\times 1)}{Z_t} + \epsilon_t$$

\vspace{2em}

\small

$$
\left[ \begin{array}
	{c} Z_t \\ \epsilon_t
\end{array}\right]
\overset{iid}{\sim} \mathcal{N}\left(
\left[ \begin{array}
	{c} 0\\ 0 
\end{array}\right],
\left[ \begin{array}
	{cc} I_k & 0\\
	0 & \Psi
\end{array}\right]\right)$$
\end{frame}
%%%%%%%%%%%%%%%%%%%%%%%%%%%%%%%%%%%%%%%%
\begin{frame}
	\frametitle{Adding Some Dynamics}
\end{frame}
%%%%%%%%%%%%%%%%%%%%%%%%%%%%%%%%%%%%%%%%

\begin{frame}
	\frametitle{Choosing the Number of Factors}
	Onatski paper: no one in the class listed it as a preference! Bai \& Ng (2002).
\end{frame}

%%%%%%%%%%%%%%%%%%%%%%%%%%%%%%%%%%%%%%%%
\begin{frame}
\frametitle{What Can We Do with Factors?}

Among other possibilities:
\begin{enumerate}
	\item Use them as Instrumental Variables 
	\item Use them to construct Forecasts
	\item Use them to ``Augment'' a VAR
\end{enumerate}

\end{frame}

%%%%%%%%%%%%%%%%%%%%%%%%%%%%%%%%%%%%%%%%
\begin{frame}[c]\frametitle{Factors as Instruments -- Bai \& Ng (2010)}
\begin{block}
   	{Endogenous Regressors $x_t$}
$$y_t = x_t' \beta + \epsilon_t \quad \quad E[x_t\epsilon_t] \neq 0 $$
\end{block}   

 \begin{block}
 	{Unobserved Variables $F_t$ are Strong IVs}
$$\underset{(k\times 1)}{x_t} = \underset{(k\times r)}{\Psi'}\underset{(r\times 1)}{F_t} + \underset{(k\times 1)}{u_t} \quad \quad E[F_t \epsilon_t] = 0$$
 \end{block}

\begin{block}
	{Observe Large Panel $z_{1t}, \hdots, z_{Nt}$}
		$$z_{it} = \lambda_i' F_t + e_{it}$$
\end{block}

\end{frame}

\end{document}