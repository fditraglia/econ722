%!TEX root = ../main.tex
\chapter{AIC-type Information Criteria}
You have probably know that Akaike's Information Criterion (AIC) summarizes the quality of a model by trading fit, measured by the maximized log likelihood, against complexity, measured by the number of estimated parameters.
But where does this complexity penalty come from?
In this chapter we'll take a closer look at the AIC and two closely criteria: Takeuichi's Information Criterion (TIC) and the ``corrected'' AIC (AIC$_c$) of Hurvich and Tsai (1989).
All three attempt to approximate the \textbf{Kullback-Leibler Divergence}, a fundamental quantity from information theory.
We'll set the stage by reviewing the main properties of the KL divergence and its connection to maximum likelihood estimation.


\section{The Kullback-Leibler Divergence}

\subsection{Basic Properties}
Suppose that $\mathbf{y}$ is a random vector drawn from a probability distribution $G$ with density $g(\mathbf{y})$. This is the \emph{true DGP} and is unknown to us. Since we don't know $g$, we attempt to approximate it using a parametric model $f(\mathbf{y}|\theta)$, where $\theta$ is a $p$-vector of parameters that we estimate via maximum likelihood.\footnote{I've written the model without covariates to keep the notation from getting out of control, but you could just as well write $f(\mathbf{y}|X,\theta)$. Similarly, I will sometimes write $f(\mathbf{y})$ for $f(\mathbf{y}|\theta)$ to simplify the notation below. } Since $f$ is not the true data density, a natural question is \emph{how well does $f$ approximate $g$}? It turns out that for maximum likelihood estimation there is a particularly convenient way to answer this question using the \textbf{Kullback Leibler Divergence}.


\begin{defn}[KL Divergence]
Let $E_G$ denote expectation with respect to the true, unknown data density $g$. Then the Kullback-Leibler divergence from $g$ to $f$ is given by
$$KL(g;f) = E_G \left[ \log{\left\{\frac{g(\textbf{y})}{f(\textbf{y})}\right\}}\right]= E_G\left[ \log{g(\textbf{y})}\right] - E_G\left[ \log{f(\textbf{y})} \right]$$
The quantity $E_G\left[ \log{f(\mathbf{y})} \right]$ is called the Expected Log-likelihood.
\end{defn}

\paragraph{Key Features of the KL Divergence}
There are several important features to note about the KL divergence:
\begin{enumerate}
\item It is \emph{not} symmetric: $KL(g;f)\neq KL(f;g)$. Hence, the KL divergence is \emph{not} a distance function (metric).
\item $KL(g;f) \geq 0$ with equality iff $f=g$. To see why, recall that, since $\log$ is a concave function, $-\log$ is convex. Thus
\begin{eqnarray*}
KL(g;f) &=& E_G \left[ \log{\left\{\frac{g(\mathbf{y})}{f(\mathbf{y})}\right\}}\right] = E_G \left[- \log{\left\{\frac{f(\mathbf{y})}{g(\mathbf{y})}\right\}}\right]\\\\
&\geq& -\log{\left\{E_G\left[ \frac{f(\mathbf{y})}{g(\mathbf{y})} \right]\right\}} = -\log{\left(\int g(\mathbf{y}) \frac{f(\mathbf{y})}{g(\mathbf{y})} \; d \mathbf{y} \right)}\\\\
&=& -\log{\left(\int f(\mathbf{y}) \;d \mathbf{y} \right)} = -\log{(1)} = 0
\end{eqnarray*}
by Jensen's Inequality. The inequality is strict only for a non-degenerate random variable and a strictly convex function. Since $-\log$ is strictly convex, the only way to make the inequality strict is for $f(\mathbf{y})/g(\mathbf{y})$ to be degenerate. This occurs precisely when $f=g$ almost everywhere.

\item Minimizing the KL divergence $KL(g;f)$ is \emph{equivalent} to maximizing the Expected Log-Likelihood $E_G[\log f(\mathbf{y})]$. This is because the first term in the KL divergence is a constant: it in no way depends on the model $f(\mathbf{y})$. The expected Log-likelihood enters negatively:
$$KL(g;f) = E_G\left[ \log{g(\textbf{y})}\right] - E_G\left[ \log{f(\textbf{y})} \right]$$
Thus, if we can find a way to estimate the Expected Log-likelihood, we can use the KL divergence for model selection: the larger the Expected Log-likelihood, the smaller the KL divergence, and the better the model.

\item The KL divergence equals the negative of \textbf{Boltzmann's Entropy} from Statistical Mechanics. Accordingly, it represents the \emph{information lost} when $g(\mathbf{y})$ is encoded by $f(\mathbf{y})$.
\end{enumerate}

\subsection{Relationship of MLE to KL}
It turns out that the KL divergence is inextricably linked to maximum likelihood estimation. To make the points a little clearer, I'll assume from now on that $\mathbf{y}$ consists of iid observations $Y_t$ for $t = 1, \hdots, T$. This is not in fact necessary for any of the derivations that follow, but it simplifies the notation. Since the expected log likelihood is unknown, we might try to approximate it using the sample analogue
$$E_{\widehat{G}}\left[\log{f(\textbf{y},\theta)} \right] = \frac{1}{T}\sum_{t=1}^T \log{f(Y_t, \theta)} = \frac{1}{T}\ell(\theta)$$
where we have replaced $G$ with the empirical distribution $\widehat{G}$. Now, by the Weak Law of Large Numbers for iid observations
$$\frac{1}{T} \ell(\theta) \overset{p}{\rightarrow} E_G\left[ \log{f(\textbf{y},\theta)} \right]$$
Under the standard regularity conditions (see Newey and McFadden, 1994) we can strengthen this result to show that
$$\hat{\theta} = \arg \max_{\theta \in \Theta} \frac{1}{T}\ell(\theta) \overset{p}{\rightarrow} \arg \max_{\theta \in \Theta} E_G\left[ \log{f(\textbf{y},\theta)}\right]$$
Since minimizing the KL divergence is the same as maximizing the expected log-likelihood we have the following result:
\begin{pro}
The ML estimator $\hat{\theta}$ converges in probability to the value of $\theta$ that minimizes the KL divergence from unknown true density $g(\mathbf{y})$ to the parametric family $f(\mathbf{y}|\theta)$. When $g(\mathbf{y})=f(\mathbf{y}|\theta)$ for some value of $\theta \in \Theta$, the divergence is minimized at zero.
\end{pro}


\subsection{A Na\"{i}ve Information Criterion}
If $g(\mathbf{y})$ were known, we could choose between two parametric models $f(\mathbf{y}|\theta)$ and $h(\mathbf{y}|\gamma)$ by comparing maximized Log-likelihoods. Define
\begin{eqnarray*}
\theta_0 &=& \arg \max_{\theta \in \Theta} E_G\left[ \log{f(\mathbf{y},\theta)} \right]\\
\gamma_0 &=& \arg \max_{\gamma\in \Gamma} E_G\left[ \log{h(\mathbf{y},\gamma)} \right]
\end{eqnarray*}
If $E_G\left[ \log{f(\mathbf{y},\theta_0)} \right] > E_G\left[ \log{h(\mathbf{y},\gamma_0)} \right]$, then the KL divergence from $g(\mathbf{y})$ to the parametric family $f_\theta$ is smaller than that from $g(\mathbf{y})$ to $h_\gamma$. Now, we know from above that $\hat{\theta} \overset{p}{\rightarrow} \theta_{0}$. Further, $\frac{1}{T}\ell(\theta) \overset{p}{\rightarrow} E_G\left[\log{f(\mathbf{y},\theta)} \right]$. Of course, $T$ will be constant across models, so why not use the maximized sample likelihood $\ell(\hat{\theta})$ for model comparison? Unfortunately, $\ell(\hat{\theta})$ is a \emph{biased estimator of the expected log likelihood} because is uses the data twice: first to estimate $\hat{\theta}$ and then directly in the sum $\sum_{t=1}^T \log{ f(Y_t,\hat{\theta})}$. Because $\hat{\theta}$ was chosen to conform to the idiosyncrasies of the data at hand, $\ell(\hat{\theta})$ is overly optimistic.

We can see this as follows. Since $\theta_0$ is the population minimizer of the KL divergence from $g$ to $f_\theta$, we have
\begin{eqnarray*}
KL\left[ g(\mathbf{y});f(\mathbf{y},\theta) \right] &\geq& KL\left[ g(\mathbf{y});f(\mathbf{y},\theta_0) \right]\\
E_G\left[ \log{g(\mathbf{y})}\right] - E_G\left[ \log{f(\mathbf{y},\theta)} \right] &\geq& E_G\left[ \log{g(\mathbf{y})}\right] - E_G\left[ \log{f(\mathbf{y},\theta_0)} \right]\\
E_G\left[ \log{f(\mathbf{y},\theta)} \right] &\leq& E_G\left[ \log{f(\mathbf{y},\theta_0)} \right]
\end{eqnarray*}
for all $\theta \in \Theta$. Recall that $\frac{1}{T}\ell(\theta) = E_{\widehat{G}}\left[ \log{f(\mathbf{y},\theta)}\right]$. By the definition of the maximum likelihood estimate, $\ell(\hat{\theta})\geq\ell(\theta_0)$. Thus,
$$E_{\widehat{G}}\left[ \log{f(\mathbf{y},\hat{\theta})}\right]\geq E_{\widehat{G}}\left[ \log{f\left(\mathbf{y},\theta_0\right)}\right] $$
In sample, the estimate $\hat{\theta}$ will show a higher maximized log-likelihood than the value of $\theta$ that maximizes the population log-likelihood. Thus, the sample analogue is \emph{overly optimistic}.

\section{The AIC and TIC}
In the previous section we saw that using the KL divergence to do model selection is equivalent to maximizing the expected log-likelihood across models. Unfortunately, using the maximized log-likelihood, based on the estimated parameters, is a biased estimator of this quantity: it is systematically too high. Both the AIC and the TIC address this problem by using asymptotic theory to get an approximate expression for the bias so that we can correct it.

To keep notation simple, throughout this section we'll assume that we have an iid sample of scalar random variables $Y_1, \hdots, Y_T$ drawn from a true but unknown distribution with density $g(y)$. As above we'll consider maximum likelihood estimation based on an approximating parametric density $f(y|\theta)$.

\subsection{Fundamental Expansion for MLE}
Under standard regularity conditions, see for example Newey and McFadden (1994), the maximum likelihood estimator $\widehat{\theta}$ can be expanded as
$$\widehat{\theta} = \theta_0 + J^{-1} \bar{U}_T + o_p(T^{-1/2})$$
where $\theta_0$ is value of $\theta$ that minimizes $KL$ divergence from $g$ to the parametric family of distributions $f(y|\theta)$ and
\begin{eqnarray*}
J &=& -E_G \left[ \frac{\partial^2 \log f(Y|\theta_0)}{\partial \theta \partial \theta'}\right]\\
\bar{U}_T &=& \frac{1}{T} \sum_{t=1}^T \frac{\partial \log f(Y_t|\theta_0)}{\partial\theta}
\end{eqnarray*}
Now, by the CLT, $\sqrt{T}\; \bar{U}_T \overset{d}{\rightarrow} U$ where $U\sim N_p(0, K)$ and
$$K = Var_G \left[ \frac{\partial \log f(Y|\theta_0)}{\partial\theta}\right] = E_G\left[ \frac{\partial \log f(Y|\theta_0)}{\partial\theta} \frac{\partial \log f(Y|\theta_0)}{\partial\theta'}\right]$$
Hence,
\begin{eqnarray*}
\sqrt{T}\left(\widehat{\theta} - \theta_0 \right) &=& \sqrt{T} \; J^{-1} \bar{U}_T + o_p(1) \\
&\overset{d}{\rightarrow}& J^{-1} U\\
&\sim& N_p(0, J^{-1}KJ^{-1})
\end{eqnarray*}
Note that when $g = f_\theta$ for some $\theta$, we have $K = J$ by the information matrix equality so the variance simplifies to $J^{-1}$.


\subsection{Estimating the Expected Log Likelihood}
To carry out model selection based on the KL divergence, we need to estimate the expected log likelihood. Under the iid assumption,
$$E_G[\log f(\mathbf{y}|\theta_0)] = E_G\left[\sum_{t=1}^T f(Y|\theta_0) \right] = T \; E_G[\log f(Y|\theta_0)]$$
so it is sufficient to work with the expected log likelihood of a single representative observation $Y$. Written as an integral,
$$E_G[\log f(Y|\theta_0)] = \int g(y) \log f(y|\theta_0) \; dy$$
There are two problems. First, we don't know $\theta_0$. Of course we do have an estimator $\widehat{\theta}$, so we might consider simply plugging it in to yield
$$\int g(y) \log f(y|\theta_0) \; dy \approx \int g(y) \log f(y|\widehat{\theta}) \; dy$$
Even with this approximation, however, we still don't know $g$, the true data density. As discussed above, trying to replace this integral with the sample analogue $\ell_T(\widehat{\theta})/T$, the maximized log-likelihood, introduces a bias. So what can we do? The idea behind the AIC and TIC is to \emph{estimate} this bias, which we'll write relative to the infeasible plug-in estimator. In other words:
$$Bias = \frac{\ell_T(\widehat{\theta})}{T} - \int g(y) \log f(y|\widehat{\theta}) dy$$
Now, as it turns out, we can expand the bias expression as follow:
\begin{eqnarray*}
Bias = \bar{Z}_T + (\widehat{\theta} - \theta_0)' J (\widehat{\theta} - \theta_0) + o_p(T^{-1})
\end{eqnarray*}
where
$$\bar{Z}_T = \frac{1}{T} \sum_{t=1}^T\left\{ \log f(Y_t|\theta_0) - E_G[\log f(Y|\theta_0)] \right\}$$
For a proof of this assertion, see Section \ref{sec:BiasExpansion}.

Now, recall that the bias expression depends on $\widehat{\theta}$ which is a random variable since it depends on the sample data. To address this, we will attempt to approximate the \emph{expectation} of the bias term, where, again, the expectation is taken over the sampling distribution of $\widehat{\theta}$. Using our asymptotic expansion:
$$E[Bias] \approx E[\bar{Z}_T] + E[(\widehat{\theta} - \theta_0)' J (\widehat{\theta} - \theta_0)]$$
Since $E[\bar{Z}_T] = 0$, this becomes
$$E[Bias] \approx E[(\widehat{\theta} - \theta_0)' J (\widehat{\theta} - \theta_0)]$$
Now, using the fundamental expansion for MLE from above
$$\sqrt{T}\left(\widehat{\theta} - \theta_0 \right) \overset{d}{\rightarrow} J^{-1}U$$
hence
$$T\left(\widehat{\theta} - \theta_0 \right)'J\left(\widehat{\theta} - \theta_0 \right) \overset{d}{\rightarrow} U' J^{-1} U$$
which suggests the approximation
$$E[Bias] \approx T^{-1} E[U'J^{-1}U]$$
Finally, using the almost magical properties of the trace operator, we have
\begin{eqnarray*}
E[U'J^{-1}U] &=& E\left[\mbox{trace} \left\{ U'J^{-1}U\right\} \right] = E\left[\mbox{trace} \left\{ J^{-1}UU'\right\} \right]\\
&=& \mbox{trace}\left\{ E[J^{-1}UU']\right\} = \mbox{trace}\left\{J^{-1} E[UU']\right\} \\
&=&\mbox{trace}\left\{ J^{-1} K\right\}
\end{eqnarray*}
Thus, we approximate the expected bias by $T^{-1}\mbox{trace}\left\{ J^{-1} K\right\}$. Finally, we correct the bias of the maximized log-likelihood and approximate the expected log likelihood by
$$E_G[\log f(Y|\theta_0)]\approx \frac{\ell(\widehat{\theta})}{T} - \frac{\mbox{trace}\left\{ J^{-1} K\right\}}{T}$$
multiplying through by $2T$ and substituting consistent estimators of the matrices $J$ and $K$ yields \textbf{Takeuchi's Information Criterion} (TIC)
$$TIC = 2\left[\ell(\widehat{\theta}) - \mbox{trace}\left\{ \widehat{J}^{-1} \widehat{K}\right\} \right]$$
The scaling is, of course, arbitrary but this particular choice is traditional. If there is a $\theta \in \Theta$ such that $g(y) = f(y|\theta)$ then the information matrix equality holds and $J^{-1} = K$. In this case $\mbox{trace}\left\{ J^{-1} K\right\} = \mbox{trace}\{\mathbf{I}_p\} = p$. Using this quantity as the bias correction yields \textbf{Akaike's Information Criterion}
$$AIC = 2 \left[\ell(\widehat{\theta}) - p \right]$$
Although the TIC and AIC are similar, there are several subtleties:
\begin{enumerate}
\item The bias correction for the AIC is derived under the assumption that the approximating model is \emph{correctly specified}, while the TIC is not.
  In this sense the AIC is a special case of the TIC.
\item It has been argued that for models where the Information Matrix Equality is not satisfied, the AIC will still be close to the TIC.
  (The log-likelihood term should dominate the bias correction in such situations.)
\item Typically, the matrices $K$ and $J$ are large, meaning that the estimates will have high variance (we need to estimate $p^2 + p$ elements). In contrast, the AIC a has much smaller variance because the bias correction \emph{does not depend on the data}. Thus, even if the model is mis-specified, it may be preferable to use AIC rather than TIC unless the sample size is large.
\end{enumerate}

\subsection{A Caveat}
To derive the TIC and AIC, we used the following expansion for the bias term
$$Bias = \bar{Z}_T + (\widehat{\theta} - \theta_0)' J (\widehat{\theta} - \theta_0) + o_p(T^{-1})$$
This holds under standard regularity conditions.
(For details on its derivation, see the next subsection.) However, we employed a bit of sleight of hand when we proceeded to approximate the expected bias using the mean of the limiting random variable $U'J^{-1}U$. For example, the expectation of ``truth'' relative to which the bias is calculated, namely
$$E_G\left[\int g(y) \log f(y|\widehat{\theta})\; dy\right]$$
does not exist in all cases. The bias correction remains reasonable in this case, as we see from the asymptotic expansion, but strictly speaking it doesn't make sense to talk about equating means.

\subsection{Appendix: Deriving the Bias Expansion}
\label{sec:BiasExpansion}
Consider a second order Taylor expansion around for $\log f(Y_t;\hat{\theta})$ around $\theta_0$:
\begin{align*}
\log f(Y_t;\hat{\theta}) &= \log f(Y_t;\theta_0)+\left.\frac{\partial\log f(Y_t;\theta)}{\partial\theta}\right|'_{\theta=\theta_0} (\widehat{\theta}-\theta_0) +  \\ 
& \quad + \frac{1}{2}(\widehat{\theta}-\theta_0)' \left.\frac{\partial^2\log f(Y_t;\theta)}{\partial\theta\partial\theta'}\right|_{\theta=\theta_0} (\widehat{\theta}-\theta_0) +R(Y_t;\hat{\theta}-\theta_0)
\end{align*}
where $R(Y_t;\hat{\theta}-\theta_0)$, the remainder, is such that:
\[
\left|R(Y_t;\widehat{\theta}-\theta_0)\right|\leq \frac{M(Y_t)}{(2+1)!}\| \widehat{\theta}-\theta_0 \|^{2+1}
\]
provided that all the derivatives of $\log f(Y_t;\theta)$ employed in the approximation are bounded above by $M(Y_t)$. The remainder has then the following property:
\[
\lim_{\widehat{\theta} \rightarrow \theta_0} \frac{R(Y_t;\widehat{\theta}-\theta_0)}{\| \widehat{\theta}-\theta_0 \|^{2}}=0\Rightarrow R(Y_t;\widehat{\theta}-\theta_0)=o(\| \widehat{\theta}-\theta_0 \|^{2})=o(1)\| \widehat{\theta}-\theta_0 \|^{2}
\]
Further implying that:
\[
R(Y_t;\hat{\theta}-\theta_0)=o_p(1)\| \hat{\theta}-\theta_0 \|^{2}=o_p(1)(\widehat{\theta}-\theta_0)'(\widehat{\theta}-\theta_0)
\]
Finally note that the $o_p(1)$ term may be a function of $Y_t$. Let's hence denote that term with $h(Y_t;\hat{\theta}-\theta_0)$ to take into account such possibility. Hence we can write:
\begin{align*}
\frac{\ell_{T}(\widehat{\theta})}{T}&= \frac{1}{T}\sum_{t=1}^{T}\left\{\log f(Y_t;\theta_0)+\left.\frac{\partial\log f(Y_t;\theta)}{\partial\theta}\right|'_{\theta=\theta_0}(\widehat{\theta}-\theta_0) + \right. \\ & \left. \quad +\frac{1}{2}(\widehat{\theta}-\theta_0)' \left.\frac{\partial^2\log f(Y_t;\theta)}{\partial\theta\partial\theta'}\right|_{\theta=\theta_0} (\widehat{\theta}-\theta_0) +h(Y_t;\hat{\theta}-\theta_0)(\widehat{\theta}-\theta_0)'(\widehat{\theta}-\theta_0) \right\}\\
& = \mathbb{E}_g \left[\log f(Y_t;\theta_0) \right]+\bar{Z}_T+\bar{U}'_T  (\widehat{\theta}-\theta_0)  - \frac{1}{2} (\widehat{\theta}-\theta_0)'  J_T  (\widehat{\theta}-\theta_0) \\
&\quad + \bar{h}(Y_t;\hat{\theta}-\theta_0)(\widehat{\theta}-\theta_0)'(\widehat{\theta}-\theta_0)
\end{align*}
with
\[
\bar{U}_T=\frac{1}{T}\sum_{t=1}^{T}\left.\frac{\partial\log f(Y_t;\theta)}{\partial\theta}\right|_{\theta=\theta_0}
\]
\[
J_T=-\frac{1}{T}\sum_{t=1}^{T}\left.\frac{\partial^2\log f(Y_t;\theta)}{\partial\theta\theta'}\right|_{\theta=\theta_0}
\] 
\[
\bar{h}(Y_t;\hat{\theta}-\theta_0)=\frac{1}{T}\sum_{t=1}^{T}h(Y_t;\hat{\theta}-\theta_0)
\]
Similarly we can write:
\begin{align*}
\int g(y)\log f(y;\widehat{\theta})dy & =  \int g(y)\left\{\log f(y;\theta_0)+\left.\frac{\partial\log f(y;\theta)}{\partial\theta}\right|'_{\theta=\theta_0}(\widehat{\theta}-\theta_0) + \right. \\ & \left. \quad +\frac{1}{2}(\widehat{\theta}-\theta_0)' \left.\frac{\partial^2\log f(y;\theta)}{\partial\theta\partial\theta'}\right|_{\theta=\theta_0} (\widehat{\theta}-\theta_0) + \right. \\
&\left. \quad+h(Y_t;\hat{\theta}-\theta_0)(\widehat{\theta}-\theta_0)'(\widehat{\theta}-\theta_0) \right\}dy \\
& = \mathbb{E}_g \left[\log f(Y_t;\theta_0) \right] - \frac{1}{2} (\widehat{\theta}-\theta_0)'  J  (\widehat{\theta}-\theta_0) + \\
& \quad + \mathbb{E}_g \left[ h(Y_t;\hat{\theta}-\theta_0) \right] (\widehat{\theta}-\theta_0)'(\widehat{\theta}-\theta_0)
\end{align*}
where the first order term drops since, by construction:
\[
\mathbb{E}_g \left[\left.\frac{\partial\log f(Y_t;\theta)}{\partial\theta}\right|_{\theta=\theta_0}\right]=0
\]
Now note that, fixing $\widehat{\theta}-\theta_0$, we have:
\[
\bar{h}(Y_t;\hat{\theta}-\theta_0) \longrightarrow_p \mathbb{E}_g \left[ h(Y_t;\hat{\theta}-\theta_0) \right] 
\]
and therefore:
\[
\bar{h}(Y_t;\hat{\theta}-\theta_0)=\mathbb{E}_g \left[ h(Y_t;\hat{\theta}-\theta_0) \right] +o_p(1)
\]
which gives us:
\begin{align*}
\bar{h}(Y_t;\hat{\theta}-\theta_0)(\widehat{\theta}-\theta_0)'(\widehat{\theta}-\theta_0)&= \mathbb{E}_g \left[ h(Y_t;\hat{\theta}-\theta_0) \right] (\widehat{\theta}-\theta_0)'  (\widehat{\theta}-\theta_0) + \\
& \quad +o_p\left(1\right)T^{-1}\sqrt{T} (\widehat{\theta}-\theta_0)' (\widehat{\theta}-\theta_0)\sqrt{T} \\
&= \mathbb{E}_g \left[ h(Y_t;\hat{\theta}-\theta_0) \right] (\widehat{\theta}-\theta_0)' (\widehat{\theta}-\theta_0) + \\
&\quad+o_p\left(1\right)T^{-1}O_p\left(1\right)^2\\
&= \mathbb{E}_g \left[ h(Y_t;\hat{\theta}-\theta_0) \right] (\widehat{\theta}-\theta_0)' (\widehat{\theta}-\theta_0) + \\
&\quad+o_p\left(T^{-1}\right)
\end{align*}
Further noting that:
\[
\text{p}\lim_{T\rightarrow\infty} J_T=J \Leftrightarrow J_T=J+o_p(1)
\]
which implies:
\begin{align*}
 (\widehat{\theta}-\theta_0)'  J_T  (\widehat{\theta}-\theta_0) &=  (\widehat{\theta}-\theta_0)'  J (\widehat{\theta}-\theta_0) +o_p\left(1\right)T^{-1}\sqrt{T}(\widehat{\theta}-\theta_0)'(\widehat{\theta}-\theta_0)\sqrt{T} \\
&= (\widehat{\theta}-\theta_0)'  J  (\widehat{\theta}-\theta_0) +o_p\left(1\right)T^{-1}O_p\left(1\right)^2\\
&= (\widehat{\theta}-\theta_0)'  J  (\widehat{\theta}-\theta_0) +o_p\left(T^{-1}\right)
\end{align*}
and recalling that:
\[
\bar{U}_T=J (\widehat{\theta}-\theta_0 )+o_p\left(T^{-1/2}\right)
\]
which similarly gives:
\begin{align*}
\bar{U}'_T (\widehat{\theta}-\theta_0)&=  (\widehat{\theta}-\theta_0)'  J  (\widehat{\theta}-\theta_0) +o_p\left(T^{-1/2}\right)T^{-1/2}\sqrt{T}(\widehat{\theta}-\theta_0) \\
&= (\widehat{\theta}-\theta_0)'  J (\widehat{\theta}-\theta_0) +o_p\left(T^{-1/2}\right)T^{-1/2}O_p\left(1\right)\\
&= (\widehat{\theta}-\theta_0)'  J  (\widehat{\theta}-\theta_0) +o_p\left(T^{-1}\right)
\end{align*}
we can finally write:
\begin{align*}
\frac{\ell_{T}(\widehat{\theta})}{T}-\int g(y)\log f(y;\widehat{\theta})dy &= \bar{Z}_T+\left(\widehat{\theta}-\theta_0\right)'  J \left(\widehat{\theta}-\theta_0\right)  + o_p\left(T^{-1}\right)
\end{align*}

\section{The Corrected AIC}
To derive the TIC and AIC we used asymptotic theory to construct an analytical bias correction. Such approximations tend to work as long as $T$ is fairly large relative to $p$ but when this is not the case, they can break down. 
We'll now consider an alternative that makes stronger assumptions and relies on \emph{exact} small-sample theory rather than asymptotics: the ``Corrected'' AIC, or AIC$_c$, of Hurvich and Tsai (1989).
Suppose that the true DGP is a linear regression model:
$$\textbf{y} = X\beta_0 + \boldsymbol{\epsilon}$$
where $\mathbf{\epsilon} \sim N(\mathbf{0}, \sigma_0^2 \mathbf{I}_T)$. 
Then $\mathbf{y}|X \sim N(X\beta_0, \sigma_0^2 \mathbf{I}_T)$ so the likelihood is
$$g(\textbf{y}|X;\beta_0, \sigma^2_0) = \left(2\pi\sigma_0^2\right)^{-T/2} \exp\left\{ -\frac{1}{2\sigma^2}(y - X\beta_0)'(y - X\beta_0)\right\}$$
and the log-likelihood is
$$\log\left[g(\textbf{y}|X;\beta_0, \sigma_0^2)\right] = -\frac{T}{2}\log(2\pi) -\frac{T}{2} \log(\sigma^2_0) - \frac{1}{2\sigma_0^2}\left(\textbf{y} - X\beta_0\right)'\left(\textbf{y} -X\beta_0\right)$$
Now suppose we evaluated the log-likelihood at some \emph{other} parameter values $\beta_1$ and $\sigma^2_1$. The vector $\beta_1$ might, for example, correspond to dropping some regressors from the model by setting their coefficients to zero, or perhaps adding in some additional regressors. 
We have
$$\log[f(\textbf{y}|X;\beta_1, \sigma_1^2)] = -\frac{T}{2}\log(2\pi) -\frac{T}{2} \log(\sigma^2_1) - \frac{1}{2\sigma_1^2}\left(\textbf{y} - X\beta_1\right)'\left(\textbf{y} -X\beta_1\right)$$
Since we've specified the density from which the data were generated as well as the density of the approximating model, we can \emph{directly calculate} the KL divergence rather than trying to find a reasonable large sample approximation. It turns out that for this example 
$$KL(g;f) = \frac{T}{2}\left[\frac{\sigma_0^2}{\sigma_1^2} - \log\left(\frac{\sigma_0^2}{\sigma_1^2}\right) - 1 \right] + \left(\frac{1}{2 \sigma_1^2}\right)\left(\beta_0 - \beta_1\right)'X'X\left(\beta_0 - \beta_1\right)$$ 
as shown in Section \ref{sec:KLderivation}.
We need to estimate this quantity for it to be of any use in model selection. If let $\widehat{\beta}$ and $\widehat{\sigma}^2$ be the maximum likelihood estimators of $\beta_1$ and $\sigma_1^2$ and substitute them into the expression for the KL divergence, we have
\begin{eqnarray*}
\widehat{KL}(g;f) &=& \frac{T}{2}\left[\frac{\sigma_0^2}{\widehat{\sigma}^2} - \log\left(\frac{\sigma_0^2}{\widehat{\sigma}^2} \right) - 1\right] + \left(\frac{1}{2\widehat{\sigma}^2} \right)\left(\beta_0 - \widehat{\beta}\right)X'X\left(\beta_0 - \widehat{\beta}\right)
\end{eqnarray*}
We still have two problems. First, we haven't been entirely clear about what $\beta_1$ and $\sigma_1$ are. At the moment, they seem to be something like ``pseudo-true'' values. Second, and more importantly, we don't know $\beta_0$ and $\sigma_0^2$ so we can't use the preceding expression to compare models.

Hurvich and Tsai (1989) address both of these problems with the assumption that all models under consideration are \emph{at least correctly specified}. That is, while they may include a regressor whose coefficient is in fact zero, they do not exclude any regressors with a non-zero coefficient. This is the same assumption that we used above to reduce TIC to AIC. Under this assumption, $\beta_1$ and $\sigma_1^2$ \emph{are precisely the same} as $\beta_0$ and $\sigma_0^2$. More importantly, we can use all of the standard results for the exact finite sample distribution of regression estimators to help us. The idea is to construct an \emph{unbiased} estimator of the KL divergence. Taking expecations and rearranging slightly, we have
\begin{eqnarray*}
E\left[\widehat{KL}(g;f) \right] &=&\frac{T}{2}\left\{ E\left[\frac{\sigma_0^2}{\widehat{\sigma}^2} \right] - \log(\sigma_0^2) + E\left[\log(\widehat{\sigma}^2)\right] -1 \right\}\\
&& \quad + \; \frac{1}{2}E\left[\left(\frac{1}{\widehat{\sigma}^2} \right)\left(\widehat{\beta}-\beta_0 \right)X'X\left(\widehat{\beta}-\beta_0\right) \right]
\end{eqnarray*}
Now, under our assumptions $T\widehat{\sigma}^2/\sigma_0^2 \sim \chi^2_{T-k}$ where $k$ is the number of estimated coefficients in $\widehat{\beta}$. Further, if $Z \sim \chi^2_\nu$ then $E[1/Z] = 1/(\nu-2)$. It follows that
$$E\left[\frac{\sigma_0^2}{\widehat{\sigma}^2} \right] = E\left[\frac{T}{T\widehat{\sigma}^2/\sigma_0^2} \right] = \frac{T}{T - k - 2}$$
We can rewrite the final term similarly:
$$E\left[\left(\frac{1}{\widehat{\sigma}^2} \right)\left(\widehat{\beta}-\beta_0 \right)X'X\left(\widehat{\beta}-\beta_0\right) \right] = E\left[\left(\frac{T}{T\widehat{\sigma}^2/\sigma_0^2} \right)\frac{\left(\widehat{\beta}-\beta_0 \right)X'X\left(\widehat{\beta}-\beta_0\right)}{\sigma_0^2} \right]$$
Under our assumptions the two terms in the product are independent, so we can break apart the expectation. First, we have
$$E\left[\frac{T}{T\widehat{\sigma}^2/\sigma_0^2} \right] = \frac{T}{T - k - 2}$$
as above. For the second part,
$$\frac{\left(\widehat{\beta}-\beta_0 \right)X'X\left(\widehat{\beta}-\beta_0\right)}{\sigma_0^2} \sim \chi^2_k$$
and hence
$$E\left[\frac{\left(\widehat{\beta}-\beta_0 \right)X'X\left(\widehat{\beta}-\beta_0\right)}{\sigma_0^2} \right] = k$$
Putting all the pieces together,
\begin{eqnarray*}
E\left[\widehat{KL}(g;f) \right] &=&\frac{T}{2}\left\{ E\left[\frac{\sigma_0^2}{\widehat{\sigma}^2} \right] + \log(\sigma_0^2) - E\left[\log(\widehat{\sigma}^2)\right] -1 \right\}\\
&& \quad + \; \frac{1}{2}E\left[\left(\frac{1}{\widehat{\sigma}^2} \right)\left(\widehat{\beta}-\beta_0 \right)X'X\left(\widehat{\beta}-\beta_0\right) \right]\\
&=& \frac{T}{2} \left( \frac{T}{T-k-2} - \log(\sigma_0^2) + E\left[\log(\widehat{\sigma}^2)\right] -1\right) + \frac{T}{2}\left(\frac{k}{T - k -2}\right)\\
&=& \frac{T}{2} \left(\frac{T + k}{T - k -2} - \log(\sigma_0^2) + E\left[\log(\widehat{\sigma}^2)\right] -1\right)
\end{eqnarray*}
Since $\log(\widehat{\sigma}^2)$ is an unbiased estimator of $E[\log(\widehat{\sigma}^2)]$, substituting this give us an unbiased estimator of $E\left[\widehat{KL}(g;f) \right]$ as desired.
The only terms that vary across candidate models are the first and the third. Moreover, the multiplicative factor of $T/2$ does not affect model selection. Hence, the criterion is
$$AIC_c = \log(\widehat{\sigma}^2) + \frac{T + k}{T - k -2}$$
In its broad strokes, this makes perfect sense. The residual error variance $\widehat{\sigma}^2$ measures in-sample fit. But since in-sample fit is a mis-leading guide to out-of-sample fit, we add a complexity penalty. Note that the way this expression is written, \emph{smaller} values indicate a better model. 

So how does this compare to the plain-vanilla AIC for normal linear regression? The maximum likelihood estimators for this problem are
\begin{eqnarray*}
\widehat{\beta} &=& (X'X)^{-1}X'\mathbf{y}\\
\widehat{\sigma}^2 &=& \frac{(\mathbf{y} - X\widehat{\beta})'(\mathbf{y} - X\widehat{\beta})}{T}
\end{eqnarray*}
It follows that the maximized log-likehood is
\begin{eqnarray*}
\log\left[f(\mathbf{y}|X;\widehat{\beta})\right] &=& -\frac{T}{2} \log(\widehat{\sigma}^2) - \frac{1}{2\widehat{\sigma}^2}(y - X\widehat{\beta})'(y -X\widehat{\beta})\\
&=& -\frac{T}{2} \log(\widehat{\sigma}^2) - \frac{T}{2}
\end{eqnarray*}
by substituting $T\widehat{\sigma}^2$ for the numerator of the second term. Hence, the AIC for this problem is
$$AIC = 2\left(\ell(\widehat{\beta}) - k \right) = -T\log(\widehat{\sigma}^2) - T - 2k $$
But this way of writing things uses the \emph{opposite} sign convention from AIC$_c$. It's important to keep track of this, since different authors use different sign conventions for information criteria. To make the AIC comparable with our scaling of the AIC$_c$, we multiply through by $-1/T$ yielding
$$AIC = \log(\widehat{\sigma}^2) + \frac{T + 2k}{T}$$
where \emph{smaller} values now indicate a better model.

\subsection{Appendix: Deriving the KL Divergence}
\label{sec:KLderivation}
  $$KL(g;f)= \int \log[g(\mathbf{y})] g(\mathbf{y}) \; d \mathbf{y} - \int \log[f(\mathbf{y})]g(\mathbf{y}) \; d \mathbf{y} = A - B$$
where
\begin{eqnarray*}
A &=&\int \left[-\frac{T}{2}\log(2\pi) -\frac{T}{2} \log(\sigma^2_0) - \frac{1}{2\sigma_0^2}\left(\textbf{y} - X\beta_0\right)'\left(\textbf{y} -X\beta_0\right) \right] g(\mathbf{y})\; d\mathbf{y}\\
&=& -\frac{T}{2}\left[ \log(2\pi) + \log(\sigma^2_0)\right] - \frac{1}{2\sigma_0^2} E_{\mathbf{y}|X}\left[\left(\textbf{y} - X\beta_0\right)'\left(\textbf{y} -X\beta_0\right)\right]\\
&=& -\frac{T}{2}\left[ \log(2\pi) + \log(\sigma^2_0)\right] - \frac{1}{2\sigma_0^2}\mbox{trace}\left\{ E_{\mathbf{y}|X}\left[\left(\textbf{y} - X\beta_0\right)\left(\textbf{y} -X\beta_0\right)'\right]\right\}\\
&=& -\frac{T}{2}\left[ \log(2\pi) + \log(\sigma^2_0)\right] - \frac{1}{2\sigma_0^2}\mbox{trace}\left\{ Var(\mathbf{y}|X)\right\}\\
&=& -\frac{T}{2}\left[ \log(2\pi) + \log(\sigma^2_0)\right] - \frac{1}{2\sigma_0^2}\left(T \sigma_0^2\right)\\
&=& -\frac{T}{2}\left[ \log(2\pi) + \log(\sigma^2_0) + 1 \right]
\end{eqnarray*}
and
\begin{eqnarray*}
B &=& \int \left[ -\frac{T}{2}\log(2\pi) -\frac{T}{2} \log(\sigma^2_1) - \frac{1}{2\sigma_1^2}\left(\textbf{y} - X\beta_1\right)'\left(\textbf{y} -X\beta_1\right) \right] g(\mathbf{y})\; d \mathbf{y}\\
&=& -\frac{T}{2}\left[\log(2\pi) + \log(\sigma^2_1) \right]- \frac{1}{2\sigma_1^2}E_{\mathbf{y}|X}\left[\left(\textbf{y} - X\beta_1\right)'\left(\textbf{y} -X\beta_1\right)\right]\\
&=& -\frac{T}{2}\left[\log(2\pi) + \log(\sigma^2_1) \right] - \left(\frac{1}{2\sigma_1^2}\right)C
\end{eqnarray*}
where we define $C$ as
\begin{eqnarray*}
C &=& E_{\mathbf{y}|X}\left[\left(\textbf{y} - X\beta_1\right)'\left(\textbf{y} -X\beta_1\right)\right]\\
&=&E_{\mathbf{y}|X}\left[\left\{ \left(\mathbf{y} - X\beta_0\right) + X\left(\beta_0 - \beta_1\right)\right\}'\left\{ \left(\mathbf{y} - X\beta_0\right)+ X\left(\beta_0 - \beta_1\right)\right\} \right]\\
&=& E_{\mathbf{y}|X}\left[\left(\mathbf{y} - X\beta_0\right)'\left(\mathbf{y} - X\beta_0\right)\right] +\; E_{\mathbf{y}|X}\left[\left(\mathbf{y} -X\beta_0\right)'X\left(\beta_0 - \beta_1\right)\right] \\
&& \quad +\; E_{\mathbf{y}|X}\left\{\left[X\left(\beta_0 -\beta_1 \right) \right]'\left(\mathbf{y} - X\beta_0 \right)\right\} + \;E_{\mathbf{y}|X}\left[\left\{X\left(\beta_0 - \beta_1\right)\right\}'\left\{X\left(\beta_0 - \beta_1\right)\right\} \right]\\
&=& Var(\mathbf{y}|X) + \; E_{\mathbf{y}|X}\left[\mathbf{y} - X\beta_0 \right]'X(\beta_0 -\beta_1) \\
&& \quad + \;(\beta_0 -\beta_1)'X'E_{\mathbf{y}|X}\left[\mathbf{y} - X\beta_0 \right] + \; (\beta_0 - \beta_1)X'X(\beta_0 - \beta_1)\\
&=& T\sigma_0^2 + 0 + 0 + (\beta_0 - \beta_1)X'X(\beta_0 - \beta_1)
\end{eqnarray*}
Hence,
\begin{eqnarray*}
B &=& -\frac{T}{2}\left[\log(2\pi) + \log(\sigma^2_1) \right] - \left(\frac{1}{2\sigma_1^2}\right) \left[ T\sigma_0^2 + (\beta_0 - \beta_1)X'X(\beta_0 - \beta_1)\right]\\
&=& -\frac{T}{2}\left[\log(2\pi) + \log(\sigma^2_1) + \frac{\sigma_0^2}{\sigma_1^2}\right] - \left(\frac{1}{2\sigma_1^2} \right)(\beta_0 - \beta_1)X'X(\beta_0 - \beta_1)
\end{eqnarray*}
and therefore,
\begin{eqnarray*}
KL(g;f) &=& A - B\\
&=& \left\{ -\frac{T}{2}\left[ \log(2\pi) + \log(\sigma^2_0) + 1 \right]\right\}\\
&& - \; \left\{ -\frac{T}{2}\left[\log(2\pi) + \log(\sigma^2_1) + \frac{\sigma_0^2}{\sigma_1^2}\right] - \left(\frac{1}{2\sigma_1^2} \right)(\beta_0 - \beta_1)X'X(\beta_0 - \beta_1) \right\}\\
&=& -\frac{T}{2} \left[ \log(\sigma_0^2) + 1 - \log(\sigma_1^2) - \frac{\sigma_0^2}{\sigma_1^2}\right] + \left(\frac{1}{2\sigma_1^2} \right)(\beta_0 - \beta_1)X'X(\beta_0 - \beta_1)\\
&=& \frac{T}{2}\left[\frac{\sigma_0^2}{\sigma_1^2} - \log\left(\frac{\sigma_0^2}{\sigma_1^2} \right) - 1\right] + \left(\frac{1}{2\sigma_1^2} \right)(\beta_0 - \beta_1)X'X(\beta_0 - \beta_1)
\end{eqnarray*}
