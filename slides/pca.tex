\section{Review of Principal Component Analysis (PCA)}
%%%%%%%%%%%%%%%%%%%%%%%%%%%%%%%%%%%%%%%%%%%%%
\begin{frame}
  \frametitle{Principal Component Analysis (PCA)}
  
  \begin{block}{Notation}
  Let $\mathbf{x}$ be a $p\times 1$ random vector with variance-covariance matrix $\Sigma$.
  \end{block}

  \begin{block}{Optimization Problem}
    \vspace{-1em}
    \[\boldsymbol{\alpha}_1 = \underset{\boldsymbol{\alpha}}{\arg \max} \; \;\mbox{Var}(\boldsymbol{\alpha}' \mathbf{x}) \quad \text{ subject to } \quad \boldsymbol{\alpha}'\boldsymbol{\alpha} = 1\]
  \end{block}

  \begin{block}{First Principal Component}
    The linear combination $\boldsymbol{\alpha}_1' \mathbf{x}$ is the \alert{first principal component} of $\mathbf{x}$.
    It is the direction along with $\mathbf{x}$ has \alert{maximal variation}
  \end{block}
\end{frame}
%%%%%%%%%%%%%%%%%%%%%%%%%%%%%%%%%%%%%%%%%%%%%
\begin{frame}
  \frametitle{Solving for $\boldsymbol{\alpha}_1$}

  \begin{block}{Lagrangian}
  $\mathcal{L}(\boldsymbol{\alpha}_1, \lambda) = \boldsymbol{\alpha}' \Sigma \boldsymbol{\alpha} - \lambda(\boldsymbol{\alpha}' \boldsymbol{\alpha} - 1)$
  \end{block}

  \begin{block}{First Order Condition}
    $2 (\Sigma\boldsymbol{\alpha}_1 - \lambda \boldsymbol{\alpha}_1) = 0 \iff (\Sigma - \lambda I_p)\boldsymbol{\alpha}_1 = 0 \iff \Sigma \boldsymbol{\alpha}_1 = \lambda \boldsymbol{\alpha}_1$
  \end{block}

  \begin{block}{Variance of 1st PC}
    $\boldsymbol{\alpha}_1$ is an e-vector of $\Sigma$ but which one? 
    Substituting, 
    \[
      \mbox{Var}(\boldsymbol{\alpha}_1' \mathbf{x}) = \boldsymbol{\alpha}_1'(\Sigma \boldsymbol{\alpha}_1) = \lambda \boldsymbol{\alpha}_1' \boldsymbol{\alpha}_1 = \lambda
    \]
  \end{block}
  \vspace{-1em}

  \begin{alertblock}{Solution}
    Var.\ of 1st PC equals $\lambda$ and this is what we want to \alert{maximize}, so $\boldsymbol{\alpha}_1$ is the e-vector corresponding to the largest e-value.
  \end{alertblock}


\end{frame}
%%%%%%%%%%%%%%%%%%%%%%%%%%%%%%%%%%%%%%%%%%%%%
\begin{frame}
  \frametitle{Subsequent Principal Components}

  \begin{block}{Additional Constraint}
    Construct 2nd PC by solving the same problem as before with the additional constraint that $\boldsymbol{\alpha}_2'\mathbf{x}$ is uncorrelated with $\boldsymbol{\alpha}_1' \mathbf{x}$.
  \end{block}

  \begin{alertblock}{$j$th Principal Component}
    The linear combination $\boldsymbol{\alpha}_j' \mathbf{x}$ where $\boldsymbol{\alpha}_j$ is the e-vector corresponding to the $j$th largest e-value of $\Sigma$.  
  \end{alertblock}
  
\end{frame}
%%%%%%%%%%%%%%%%%%%%%%%%%%%%%%%%%%%%%%%%%%%%%
\begin{frame}
  \frametitle{Sample PCA}

  \begin{block}{Notation}
    $X = (n\times p)$ \alert{centered} data matrix -- columns are mean zero.
  \end{block}

  \begin{block}{SVD}
   $X = UDV'$, thus $X'X = VDU'UDV' = VD^2V'$
  \end{block}

  \begin{block}{Sample Variance Matrix}
    $S = n^{-1}X'X$ has same e-vectors as $X'X$ --  the columns of $V$! 
  \end{block}

  \begin{block}{Sample PCA}
    Let $\mathbf{v}_j$ be the jth column of $V$. Then,
    \begin{eqnarray*}
      \mathbf{v}_j &=& \text{PC loadings for jth PC of } S\\
      \mathbf{v}_j' \mathbf{x}_i &=& \text{PC score for individual/time period } i \\
    \end{eqnarray*}
  \end{block}
\end{frame}
%%%%%%%%%%%%%%%%%%%%%%%%%%%%%%%%%%%%%%%%%%%%%
\begin{frame}
  \frametitle{Sample PCA}
  \small

  \begin{block}{PC scores for $j$th PC}
    \vspace{-0.5em}
    \[
      \mathbf{z}_j = 
      \left[
        \begin{array}{c}
          z_{j1}\\ \vdots \\ z_{jn}
      \end{array}
    \right] =
      \left[
        \begin{array}{c}
          \mathbf{v}_j'\mathbf{x}_1\\
          \vdots\\
          \mathbf{v}_j'\mathbf{x}_n
      \end{array}
    \right] = 
      \left[
        \begin{array}{c}
          \mathbf{x}_1'\mathbf{v}_j\\
          \vdots\\
          \mathbf{x}_n'\mathbf{v}_j
      \end{array}
    \right] = 
      \left[
        \begin{array}{c}
          \mathbf{x}_1'\\
          \vdots\\
          \mathbf{x}_n'
      \end{array}
    \right]\mathbf{v}_j = X\mathbf{v}_j
    \]
  \end{block}

  \begin{block}{Getting PC Scores from SVD}
   Since $X = UDV'$ and $V'V = I$, $XV = UD$, i.e.\

   \[
     \left[
     \begin{array}{c}
       \mathbf{x}_1'\\
       \vdots\\
       \mathbf{x}_n'
     \end{array}
   \right]
   \left[
   \begin{array}{ccc}
     \mathbf{v}_i & \cdots & \mathbf{v}_p
   \end{array}
 \right] = 
 \left[
 \begin{array}{ccc}
   \mathbf{u}_1 & \cdots & \mathbf{u}_r
 \end{array}
 \right]
 \left[
 \begin{array}{ccc}
   d_1 & \cdots & 0\\
   & \ddots  & \\
   0 & \cdots & d_r
 \end{array}
 \right]
   \]

   \vspace{0.5em}

   \alert{Hence we see that $\mathbf{z}_j = d_j \mathbf{u}_j$}

  \end{block}

  
\end{frame}
%%%%%%%%%%%%%%%%%%%%%%%%%%%%%%%%%%%%%%%%%%%%%
\begin{frame}
  \frametitle{Properties of PC Scores $\mathbf{z}_j$}

  Since $X$ has been de-meaned:
  \[
    \bar{z}_j = \frac{1}{n}\sum_{i=1}^n \mathbf{v}_j'\mathbf{x}_i = \mathbf{v}_j' \left( \frac{1}{n}\sum_{i=1}^n \mathbf{x}_i \right) = \mathbf{v}_j' \mathbf{0} = 0
  \]

  Hence, since $X'X = VD^2V'$
  \[
    \frac{1}{n}\sum_{i=1}^n (z_{ji} - \bar{z}_j)^2 = \frac{1}{n} \sum_{i=1}^n z_{ji}^2 = \frac{1}{n} \mathbf{z}_j'\mathbf{z}_j = \frac{1}{n}\left( X\mathbf{v}_j \right)'\left( X\mathbf{v}_j \right) = \mathbf{v}_j' S\mathbf{v}_j = d_j^2/n
  \]

  
\end{frame}
%%%%%%%%%%%%%%%%%%%%%%%%%%%%%%%%%%%%%%%%%%%%%
\section{Principal Components Regression}
%%%%%%%%%%%%%%%%%%%%%%%%%%%%%%%%%%%%%%%%%%
\begin{frame}
  \frametitle{Principal Components Regression (PCR)}

  \small

Instead of ``smooth weights'' as in Ridge, truncate the PCs:
	\begin{enumerate}
    \item Calculate SVD $X=UDV'$ of \alert{centered} data matrix $X$
		\item Construct the sample principal components: $\mathbf{z}_i = d_j \mathbf{u}_j$.
		\item Throw away all but first $k$ principal components, where $k <p$.
		\item Regress \textbf{y} on $\mathbf{z}_1, \hdots, \mathbf{z}_M$. 
	\end{enumerate}

\end{frame}
%%%%%%%%%%%%%%%%%%%%%%%%%%%%%%%%%%%%%%%%%%
\begin{frame}
  \frametitle{PCR versus Ridge}
  \begin{itemize}
    \item PCR is a much less smooth version of Ridge
    \item Conventional wisdom is that PCR will perform worse since it shrinks low variance directions too much and doesn't shrink high variance directions at all.
  \item However, Dhillon et al.\ (2013) show that the MSE risk of PCR is always within a constant factor of that of Ridge Regression while there are situations in which Ridge can be arbitrarily worse than PCR in terms of MSE. 
    \item In practice, which is better depends on the DGP
  \end{itemize}
\end{frame}
%%%%%%%%%%%%%%%%%%%%%%%%%%%%%%%%%%%%%%%%%%
