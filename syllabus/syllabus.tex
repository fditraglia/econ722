\documentclass[11pt, letterpaper]{article}
\usepackage{geometry}
\geometry{margin=1in} 
\usepackage{setspace}
\linespread{1}
\usepackage{hyperref}
\usepackage{totcount}
\usepackage{termcal}
\usepackage{enumerate}
\usepackage{fancybox}
\usepackage{amsmath, amssymb}

%%%%%%%%%%%%% A Command to create an automatically numbered Quiz Icon in the course calendar.
\newtotcounter{quizcounter}
\newcommand{\quiz}{\addtocounter{quizcounter}{1}\fbox{\textbf{Quiz \#\arabic{quizcounter}}}}


% Some useful commands (our classes always meet either on Monday and Wednesday 
% or on Tuesday and Thursday)
\newcommand{\MWClass}{%
\calday[Monday]{\classday} % Monday
\skipday % Tuesday (no class)
\calday[Wednesday]{\classday} % Wednesday
\skipday % Thursday (no class)
\skipday % Friday 
\skipday\skipday % weekend (no class)
}

\newcommand{\TRClass}{%
\skipday % Monday (no class)
\calday[Tuesday]{\classday} % Tuesday
\skipday % Wednesday (no class)
\calday[Thursday]{\classday} % Thursday
\skipday % Friday 
\skipday\skipday % weekend (no class)
}

\newcommand{\Holiday}[2]{%
\options{#1}{\noclassday}
\caltext{#1}{#2}
}

\begin{document}


\thispagestyle{plain}

\begin{center}
\Large
\sc
Econometrics IV\\
\large
Econ 722\\
\large
Spring 2019
\end{center}



\normalsize

\noindent \textbf{Course Instructor:} Francis DiTraglia \\
Office: PCPSE 630\\
Office Hours: M 3--4pm, R 4--5pm 

\medskip

 
\noindent \textbf{Lecture Time and Location:} 
TR 1:30--3PM PCPSE 203

\medskip
 
\noindent \textbf{Website and Courseware:} Lecture slides and notes, as well as problem sets and a copy of this syllabus are available at \url{http://ditraglia.com/econ722}.
Course announcements will be posted on Canvas: \url{http://canvas.upenn.edu}.
Please submit all assignments electronically via Canvas rather than handing in paper copies.
Course readings will be shared via Dropbox.

\medskip



\noindent \textbf{Course Description:} 
This course covers statistical decision theory, model selection, moment selection and averaging, high-dimensional regression, factor models, and selective inference.
See the semester calendar at the end of this document for more details.
Topics may vary depending on time constraints and the interests of the students registered for the course.
We will also discuss effective writing and presentation for economics and econometrics.

\medskip

\noindent \textbf{Prerequisites:} 
Econ 705 and 706 or equivalent graduate level econometrics. 



\medskip

\noindent \textbf{Readings:} 
The readings for this course include my lecture slides and notes, available on the course website, along with references that I will share through a password protected Dropbox folder.
A recommended text for the machine learning material in the course is \emph{Machine Learning: A Probabilistic Perspective} by Kevin Murphy (2012).

\medskip

\section*{Assignments and Grading}
As this course comes at the very end of your second year, I see no point in wasting your time with endless problem sets.
Thus, while I will require you to complete some problems related to the course material, the bulk of the course assignments are intended to help you begin work on your third year paper, whether you plan to specialize in econometrics or another field.
	\begin{equation*}
    \begin{split}
    \mbox{Grade} = (10\% \text{ Participation}) + (10\% \text{ Problem Presentation}) + (20\% \text{ Problem Set})\\
    + (10\% \text{ Paper Idea}) + (20\% \text{ Referee Report}) + (30\% \text{ Final Presentation})
    \end{split}
	\end{equation*}

\noindent \textbf{Participation:} 
I expect you to attend each class meeting and participate actively.
This means arriving on time, paying attention, and taking part in class discussions.
In some cases I will ask you to complete a reading assignment before class and come prepared to discuss.
I expect you to take such assignments seriously: they are just as much a part of the course as the homework and other assignments.
If you follow these guidelines in good faith, you will receive full credit for participation.
If you do not, I will deduct points as the situation warrants.

\medskip

\noindent \textbf{Problem Set:} 
Your homework grade will have two components, each of which depends on a set of fourteen homework problems that I have posted on the course website.
The first component is a problem set worth 20\% of your course grade, and due at 11:59pm on May 1st, 2019.
For your problem set you must submit solutions to six of the fourteen posted problems: four from the section marked ``Theoretical'' and two from the section marked ``Computational.''
Subject to this constraint, you may choose any problems that you like. 
Your full set of solutions must be type-written, and submitted in pdf form to
canvas. 
Along with your pdf write-up, make sure to submit all the source code required to replicate your solutions to the computational exercises. 
This code must be written in an open-source language. 
R, Python, Julia, and Octave are permitted; Matlab is not.
While your problem set grade will depend on the completeness and correctness of your answers, clarity of writing and exposition are equally important. 

\medskip

\noindent \textbf{Problem Presentation:}
The second component of your homework grade is a problem presentation worth 10\% of your course grade.
Each student will present a solution to one homework problem during the semester.
This presentation will take place during class time and should last between 5 and 7 minutes.
Afterwards, we will discuss both the substance and style of your presentation.
You may use chalk, slides, or a combination of both.
The appropriate choice will depend on the question.
The problem you present, along with the date of your presentation, must be agreed in advance with the instructor.

\medskip

\noindent \textbf{Paper Idea:}
Each student must submit an idea for a third-year paper topic to canvas by 11:59pm on Sunday, March 31st.
Your submission should be in pdf form, and no more than two pages in length.
The key word is \emph{idea}: I do not expect a fully fleshed-out research proposal or a detailed literature review.
Take this as an opportunity to think about the questions and topics that most excite you and convey this excitement in your writing.

\medskip
\noindent \textbf{Referee Report:} 
Each student must submit a referee report to canvas in pdf form by 11:59pm on May 1st. 
Your report should be 3--5 pages in length and discuss a recent working paper agreed in advance with the instructor.
I will provide guidelines for referee reports in class.

\medskip 
\noindent \textbf{Final Presentation:} 
Each student will give an in-class presentation, with slides, of approximately 20 minutes in length.
Presentations will take place during the final two class meetings of the semester, and the topic must be agreed with the instructor in advance.
You have three options for your presentation: (1) a literature review and research proposal, (2) a summary of an important paper, or (3) an overview of an econometric method that we did not cover in Econ 722.
If you elect to present a paper, it must be different from the one you selected for your referee report.
Whichever option you choose, the content of your presentation should relate in some way to your third-year paper idea.



\newpage

%NOTE: don't use leading zeros in dates! In other words, use 1/1/2014 rather than 01/01/2014

\section*{Semester Calendar}

\begin{center}
\small
\begin{calendar}{3/11/2019}{8} %Date of Monday in first week of classes, NOT the date of the first class!
\setlength{\calboxdepth}{.25in}
\TRClass
% schedule
%\caltexton{1}{Computing for Econometrics: Creating R Packages, Rcpp, Parallel R}
\caltexton{1}{Decision Theory I}
\caltextnext{Decision Theory II, Writing Workshop}
\caltextnext{Model Selection I: AIC, TIC, Corrected AIC, Mallow's $C_p$}
\caltextnext{Model Selection II: Bayesian Model Comparison, BIC, Cross-validation}
\caltextnext{Model Selection III: Asymptotic Properties, Consistent vs.\ Efficient Model Selection}
\caltextnext{Moment Selection I: Consistent Moment Selection}
\caltextnext{Presentation Workshop -- \textbf{(2pm Start time!)}}
\caltextnext{Moment Selection II: Focused Moment Selection and Averaging, Inference} 
\caltextnext{High-Dimensional Regression I: QR and SV Decompositions, PCA, Ridge Regression}
\caltextnext{High-Dimensional Regression II: LASSO, PCR}
\caltextnext{High-Dimensional Regression III}
\caltextnext{Selective Inference}
\caltextnext{Student Presentations I}
\caltextnext{Student Presentations II}
% ... and so on

% Holidays
\Holiday{3/12/2019}{\textbf{No Class}}
\Holiday{5/2/2019}{\textbf{No Class}}

\end{calendar}
\end{center}


\end{document}
