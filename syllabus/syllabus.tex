\documentclass[11pt, letterpaper]{article}
\usepackage{geometry}
\geometry{margin=1in} 
\usepackage{setspace}
\linespread{1}
\usepackage{hyperref}
\usepackage{totcount}
\usepackage{termcal}
\usepackage{enumerate}
\usepackage{fancybox}
\usepackage{amsmath, amssymb}

%%%%%%%%%%%%% A Command to create an automatically numbered Quiz Icon in the course calendar.
\newtotcounter{quizcounter}
\newcommand{\quiz}{\addtocounter{quizcounter}{1}\fbox{\textbf{Quiz \#\arabic{quizcounter}}}}


% Some useful commands (our classes always meet either on Monday and Wednesday 
% or on Tuesday and Thursday)
\newcommand{\MWClass}{%
\calday[Monday]{\classday} % Monday
\skipday % Tuesday (no class)
\calday[Wednesday]{\classday} % Wednesday
\skipday % Thursday (no class)
\skipday % Friday 
\skipday\skipday % weekend (no class)
}

\newcommand{\TRClass}{%
\skipday % Monday (no class)
\calday[Tuesday]{\classday} % Tuesday
\skipday % Wednesday (no class)
\calday[Thursday]{\classday} % Thursday
\skipday % Friday 
\skipday\skipday % weekend (no class)
}

\newcommand{\Holiday}[2]{%
\options{#1}{\noclassday}
\caltext{#1}{#2}
}

\begin{document}


\thispagestyle{plain}

\begin{center}
\Large
\sc
Econometrics IV\\
\large
Econ 722\\
\large
Spring 2019
\end{center}



\normalsize

\noindent \textbf{Course Instructor:} Francis DiTraglia \\
Office: PCPSE 630\\
Office Hours: M 3--4pm, R 4--5pm 

\medskip

 
\noindent \textbf{Lecture Time and Location:} 
TR 1:30--3PM PCPSE 203

\medskip
 
\noindent \textbf{Website and Courseware:} Lecture slides and notes, as well as problem sets and a copy of this syllabus will be posted at \url{http://ditraglia.com/econ722}.
Course announcements will be posted on Canvas: \url{http://canvas.upenn.edu}.
Please submit your problem sets electronically via Canvas rather than handing in paper copies.
Course readings will be shared via Dropbox.

\medskip



\noindent \textbf{Course Description:} 
This course will cover statistical decision theory, model selection, moment selection and averaging, high-dimensional regression (including Ridge, LASSO, and PCR), high-dimensional factor models, and selective inference.
Topics may vary somewhat depending on time constraints and the interests of the students registered for the course.
See the semester calendar at the end of this document for more details.
We will also spend some class time discussing effective writing and presentation for economics, with a particular focus on presenting technical and quantitative information.

\medskip

\noindent \textbf{Prerequisites:} 
Econ 705 and 706 or equivalent graduate level econometrics. 



\medskip

\noindent \textbf{Readings:} 
The readings for this course will include my lecture slides and notes, as well as research papers that I will share with you via a password protected Dropbox folder.
The recommended text for this course is \emph{Machine Learning: A Probabilistic Perspective} by Kevin Murphy (2012).

\medskip

\section*{Assignments and Grading}
	\begin{equation*}
	\begin{split}
    \mbox{Grade} = (25\% \times \mbox{Referee Report}) + (25\% \times \mbox{Presentation})  + (50\% \times \mbox{Problem Sets \& Participation}) 
	\end{split}
	\end{equation*}


\medskip

\noindent \textbf{Referee Report:} 
Each student will be assigned a recent working paper for which to write a referee report between 3--5 pages in length.
The report should provide a detailed summary of the paper, along with specific comments on how it could be improved.
Details will be discussed in class.

\medskip 
\noindent \textbf{Presentation:} 
Each student will give an in-class presentation, with slides, of approximately 20 minutes in length.
Presentations will take place in the second half of April.
The topic of the presentation must be related to the course material and agreed upon with the instructor in advance. 
Your presentation may take the form of either (1) a literature review and proposal for a research paper or (2) a summary of a paper or method related to but not covered in Econ 722.
If you elect to present a paper, it must be different from the one you were assigned for your referee report. 
As you will be graded on both the quality of your content and your presentation skills, I will spend some time in class discussing how to give a good talk and provide you with relevant readings. 

\medskip 
\noindent \textbf{Problem Sets and Class Participation:} 
I will assign four problem sets during the semester.
You are also expected to study the assigned readings, attend class, and participate in discussions.





%NOTE: don't use leading zeros in dates! In other words, use 1/1/2014 rather than 01/01/2014

\section*{Semester Calendar}

\begin{center}
\small
\begin{calendar}{3/11/2019}{8} %Date of Monday in first week of classes, NOT the date of the first class!
\setlength{\calboxdepth}{.25in}
\TRClass
% schedule
%\caltexton{1}{Computing for Econometrics: Creating R Packages, Rcpp, Parallel R}
\caltexton{1}{Decision Theory I}
\caltextnext{Decision Theory II, Writing Workshop}
\caltextnext{Model Selection I: AIC, TIC, Corrected AIC, Mallow's $C_p$}
\caltextnext{Model Selection II: Bayesian Model Comparison, BIC, Cross-validation}
\caltextnext{Model Selection III: Asymptotic Properties, Consistent vs.\ Efficient Model Selection}
\caltextnext{Moment Selection I: Consistent Moment Selection}
\caltextnext{Presentation Workshop -- \textbf{(2pm Start time!)}}
\caltextnext{Moment Selection II: Focused Moment Selection and Averaging, Inference} 
\caltextnext{High-Dimensional Regression I: QR and SV Decompositions, PCA, Ridge Regression}
\caltextnext{High-Dimensional Regression II: LASSO, PCR}
\caltextnext{High-Dimensional Regression III}
\caltextnext{Selective Inference}
\caltextnext{Student Presentations I}
\caltextnext{Student Presentations II}
% ... and so on

% Holidays
\Holiday{3/12/2019}{\textbf{No Lecture}}

\end{calendar}
\end{center}


\end{document}
